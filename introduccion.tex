

El sistema descrito en \'este manual de usuario tiene como finalidad satisfacer las necesidades de los usuarios antes mencionados dentro del \'ambito laboral, as\'i como solventar las problem\'aticas identificadas en el documento de An\'alisis de forma eficaz y eficiente. Por mencionar algunas de las funciones de \'este software tenemos el registro de expedientes m\'edicos, generaci\'on de recetas, registro de pago de medicamentos, entre otras.
\'Este documento fue elaborado por EquipoUNO, conformado por los siguientes integrantes:
\begin{itemize}
\item Cabello Acosta Gerardo Aramis
\item Barrios Alvarado Daniel Alejandro
\item Maldonado Ledo Diana Guadalupe
\item Cuellar Sanchez Ricardo
\end{itemize}

%--------------------------------------------------
\section{Propósito}
El prop\'osito de \'este documento es explicar con detalle una serie de cosas como son:
\begin{itemize}
\item Los requisitos necesarios para usar el sistema.
\item C\'omo instalar y/o configurar el sistema.
\item La forma de usar todas las funciones que el sistema posee.
\item Describir posibles problemas que se puedan presentar y qu\'e hacer en caso de ocurrir uno.
\item Definici\'on de tecnicismos para el correcto entendimiento del documento.
\item Mostrar y responder preguntas frecuentes relacionadas al sistema.

\end{itemize}

%--------------------------------------------------

